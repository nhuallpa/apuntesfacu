\documentclass{article}
\usepackage{amsmath}
\usepackage{graphicx}
\usepackage{listings}
\usepackage{xcolor}
\usepackage[utf8]{inputenc}


\definecolor{codegreen}{rgb}{0,0.6,0}
\definecolor{codegray}{rgb}{0.5,0.5,0.5}
\definecolor{codepurple}{rgb}{0.58,0,0.82}
\definecolor{backcolour}{rgb}{0.95,0.95,0.92}
\usepackage[top=3cm, bottom=3cm, right=3cm, left=3cm]{geometry}

\lstdefinestyle{mystyle}{
    backgroundcolor=\color{backcolour},   
    commentstyle=\color{codegreen},
    keywordstyle=\color{magenta},
    numberstyle=\tiny\color{codegray},
    stringstyle=\color{codepurple},
    basicstyle=\ttfamily\footnotesize,
    breakatwhitespace=false,         
    breaklines=true,                 
    captionpos=b,                    
    keepspaces=true,                 
    numbers=left,                    
    numbersep=5pt,                  
    showspaces=false,                
    showstringspaces=false,
    showtabs=false,                  
    tabsize=2
}

\lstset{style=mystyle, texcl=true}

\begin{document}

\newpage{}
\tableofcontents
\newpage{}

\newpage
\section{Semaforos}

\section{Problemas tipicos con Semaforos}

\subsection{Productor consumidor - buffer infinito}

\lstinputlisting[language=Ada]{codigo/AlgoritmoProductorConsumidor_BufferInfinito.ads}

\newpage
\subsection{Productor consumidor - buffer finito}

Invariantes:
\begin{itemize}
    \item No se puede consumir lo que no hay.
    \item Todos los item producidos son eventualmente consumidos.
    \item Al espacio de almacenamiento se accede de a uno.
    \item Se debe respetar el orden de almacenamiento y retiro de los elementos.
\end{itemize}

El buffer sirve para comunicar dos componentes que colocan y sacan a velocidades parecidas.
Hay dos problemas de sincronismo.
\begin{enumerate}
    \item No se puede consumir si el buffer esta vacio
    \item No se puede producir si el buffer esta lleno. 
\end{enumerate}

\lstinputlisting[language=Ada]{codigo/AlgoritmoProductorConsumidor_BufferFinito.ads}
\begin{quote}
    La lectura es destructiva y La soluciones es simetrica, ambos tienen el mismo codigo.        
\end{quote}

\newpage
\subsection{Lectores escritores}

Para situaciones donde las estructuras de datos, base de datos o sistema de archivos son leidos y escritos por hilos concurrentes.

La solución es asimetrica. El escritor y el productor ejecutan códigos distintos antes de entrar en la seccion critica.

Las restricciones de sincronizacion son:
\begin{itemize}
    \item Readers : Es proceso no requiere la exclución de otro proceso Readers.
    \item Writers : Proceso que requiere la exclución de otros procesos Reader y Writers
\end{itemize}

El patron de exclución se llama Categorical Mutual Exclution.
Un hilo en la sección critica no necesita excluir otro hilos pero en la presencia de una categoria en la seccion critica excluye otras categorias.

\subsubsection{SOLUCION 1: Prioridad a Lectores}

\lstinputlisting[language=Ada]{codigo/AlgoritmoLectoresEscritore_Semaforos1.ads}

\newpage
\subsubsection{SOLUCION 1a: Usando Lightswich}
Se utiliza un patron llamado Lightswich. El primero que entra prende la luz, el último que sale lo apaga. Entonces se arma un pre y pos protocolo para simplificar el reader.

\lstinputlisting[language=Ada]{codigo/AlgoritmoLectoresEscritore_Semaforos1a.ads}

\textbf{PROBLEMA}: Es posible que el writer tenga stavation.
Si un escritor llega mientras hay escritores en la sección critica, este puede esperar en la cola por siempre mientras que los lector continuan llegando. 
Mientras llegen lectores ante del último de los lectores, siempre habra un lector en el Room. Hasta que los lectores no llegen a CERO, no pueden dale paso a los escritores.

\newpage
\subsubsection{SOLUCION 2: Agregar un molinete (turnstile).}

\lstinputlisting[language=Ada]{codigo/AlgoritmoLectoresEscritore_Semaforos2.ads}

Esta solución garantiza que al menos un escritor proceda. En esta solución no hay prioridad entre escritores y lectores. Hay Justicia!. Pero aun es posible que lectores entran mientras haya mas escritores encolados.

A veces es preferible una solución que da mas prioridad a los escritores! Y esto depende del contexto.
Por ejemplo: Si los writes tienen tiempos criticos de updates, es mejor minimizar las lecturas de datos viejos y priorizar la actualización.

\newpage
\subsubsection{SOLUCION 3: Mas prioridad a los escritores}

Mientras entran escritores, ningún lector debe entrar hasta que se haya ido el último escritor.

\lstinputlisting[language=Ada]{codigo/AlgoritmoLectoresEscritore_Semaforos3.ads}

\newpage
\subsection{Filosofos comenzales}
\begin{itemize}
    \item Acciones Pensar y comer
    \item 1 Mesa con 5 platos, 5 tenedores
\end{itemize}

Sincronización:
\begin{itemize}
    \item Cada filosofo puede tomar un tenedor de la izquierda y de la derecha, pero solo uno a la vez.
    \item Un filosofo solo puede comer si tiene dos tenedores.
\end{itemize}

Restricciones:
\begin{itemize}
    \item Un filosofo solo puede comer si tiene dos tenedores.
    \item Dos filosofos no pueden tener el mismo tenedor simultaneamente
    \item No hay deadlock
    \item No hay indivual stavation
    \item Comportamiento eficiente bajo ausencia de contención.
\end{itemize}

\subsubsection{SOLUCIÓN 1}

\lstinputlisting[language=Ada]{codigo/AlgoritmoFilosofosComensales_Semaforos1.ads}

\begin{quote}
    Presenta deadlock porque los 5 filosofos pueden tomar el tener de la izq al mismo tiempo.
\end{quote}

\newpage
\subsubsection{SOLUCIÓN 2}

\lstinputlisting[language=Ada]{codigo/AlgoritmoFilosofosComensales_Semaforos2.ads}

Se asumio que el semaforo Room es de tipo Cola-Bloqueante, entonces cada filosofo entrar al cuarto eventualmente. Con esto nos aseguramos que haya justicia.
El resto de los semaforos, pueden ser block-set.

\subsubsection{SOLUCIÓN 3: SOLUCION ASIMETRICA}

Los primeros 4 filosofos ejecutan la solución 1, pero el 5to filosofo (cualquier), toma primero el de la derecha y luego el de la izquierda.

\lstinputlisting[language=Ada]{codigo/AlgoritmoFilosofosComensales_Semaforos3.ads}

\begin{quote}
    De esta manera nos evitamos entrar en deadlock o starvation
\end{quote}

\newpage
\subsection{Panadero}
Es un algoritmo de exclución mutua para N procesos.

Un proceso que desea entrar en la seccion critica, se requiere que tome un numero de ticket el cual sera el valor mas grande de todos lo que hayan tomado un ticket previamente. 
Situacion similar a la de los clientes entran a una panaderia y toman un ticket para ser atendidos.

\subsubsection{VERSION SIMPLIFICADA: para dos procesos}

\lstinputlisting[language=Ada]{codigo/AlgoritmoBakery_2.ads}

\subsubsection{VERSION N PROCESOS}

\lstinputlisting[language=Ada]{codigo/AlgoritmoBakery_N.ads}


Conclución, no es práctico por dos razones.
\begin{enumerate}
    \item El número de ticket no tiene limite si algún proceso siempre esta en la zona critica.
    \item Cada proceso tiene que consultar el número de ticket del resto de los procesos.
\end{enumerate}

\section{Monitores}

\section{Problemas tipicos con monitores}


\end{document}

