\documentclass{article}

\begin{document}

Jerarquia de Memoria
--------------------

La Jerarquia de Memoria, toma las ventajas del principio de localidad y el costo de performance de las memorias.
El principio de localidad dice que los programas no acceden al todo el codigo o datos uniformemente.

Tipico nivel de herarquia:

    CPI   |   Cache   |  Memoria  | I/O device


Como la memorias rapidas son caras, la Jerarquia de Memoria esta organizada en distitos niveles.

El objetivo es proveer un sistema de memoria con costos tan bajos como el nivel de memoria mas barato y
velocidad casi tan tapidos como el nivel mas veloz.

Review Caches
-------------

Cache: Es el primer nivel de la Jerarquia. 
    - Se utiliza cuando sea necesario un buffer para reusar items que ocurren comunmente.

Cache Hit: Cuando la CPU requiere un dato y lo cuentra como item en la cache.
Cache Miss: Cuando la CPU requiere un dato y NO lo cuentra como item en la cache.
    - El tiempo requerido depende de la latencia y el ancho de banda de la memoria.
    - Es manejado por el hardware.

Bloque: Una coleccion de datos con tamaño fijo que contiene la palabra(dato) requerida, es recuperada
de memoria y situada en la cache.

Localidad temporal: Nos dice que podemos probablemente vamos a necesitar el mismo datos en un futuro cercano.
    Entonces es util dejarlo en la cache para utilizarlo rapidamente.

Localidad espacial: Nos dice que hay altas posibilidades de utilizar pronto otro dato del mismo bloque.

Paginas: Espacios de memoria separados en bloques de tamaño fijo. 
    - Una pagina puede recidir tanto en memoria como en disco.
    - Cuando el CPU hace referencia a un item dentro de una pagina que no esta presente (not present)
      en la cache, or memoria principal, se produce un PAGE FAULT, y la pagina entera es movida
      del disco a memoria principal.
      Como los page fault toman mucho tiempo, estos se manejan por software y la CPU no se detiene(STALLED)
      La cache y la memoria principal tienen la misma relacion en memoria y en disco.


Perfomance de Cache:
-------------------
    Para evaluar la perfomance de la cache expandiremos el uso de la ecuación de CPU Excecution Time.

    Memory Stall Cycles: Numero de ciclos que necesita esperar la CPU para acceder a la memoria.

    CPU exe = (CPU clock cycles + Memory Stall cycles ) * Clock cycle time.
    
    Memory Stall cycles depende de la cantidad de misses y el costo por miss.

    \begin{equation}
        Memory Stall cycles = IC * Misses/Instruction * Miss Penalty
    \end{equation}

        Donde IC es Instruction Count. 


\end{document}
