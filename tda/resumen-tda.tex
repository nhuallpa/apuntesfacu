\documentclass{article}
\usepackage{amsmath,amssymb}
\usepackage{graphicx}
\usepackage[top=3cm, bottom=3cm, right=3cm, left=3cm]{geometry}
\usepackage{xcolor}
\usepackage{listings}

\definecolor{codegreen}{rgb}{0,0.6,0}
\definecolor{codegray}{rgb}{0.5,0.5,0.5}
\definecolor{codepurple}{rgb}{0.58,0,0.82}
\definecolor{backcolour}{rgb}{0.95,0.95,0.92}

\lstdefinestyle{mystyle}{
    backgroundcolor=\color{backcolour},   
    commentstyle=\color{codegreen},
    keywordstyle=\color{magenta},
    numberstyle=\tiny\color{codegray},
    basicstyle=\ttfamily\footnotesize,
    breakatwhitespace=false,         
    breaklines=true,                 
    captionpos=b,                    
    keepspaces=true,                 
    numbers=left,                    
    numbersep=5pt,                  
    showspaces=false,                
    showstringspaces=false,
    showtabs=false,                  
    tabsize=2
}

\lstset{style=mystyle}

\begin{document}

\newpage{}
\tableofcontents
\newpage{}

\newpage
\section{Stable Maching problema}

\subsection{Algoritmo Gale-Shapley}
Este algoritmo al terminar de ejecutarse se encuentra un matching prefecto si:
\begin{itemize}
    \item Si existen \(n\) solicitantes con diferentes preferencias.
    \item Si existen \(n\) requeridos con diferentes preferencias.
\end{itemize}
Eligiendo las estructuras correctamente se puede plantear en \(O(n)\).

\begin{lstlisting}[language=Python, caption=Algoritmo de Gale-Shapley]
    Inicialmente M=Vacio
    
    Mientras existe un solicitante sin pareja que no aun se haya postulado a todas las parejas
    
        Sea s un solicitante sin pareja
        Sea r el requerido de su mayor preferencia al que no le
                    solicito previamente
            
        if r esta desocupado
            M = M U (s,r)
            s esta ocupado
        else
            Sea s' tal que (s', r) pertenece a M

            si r prefiere a s sobres s'
                M = M - {(s', r)} U (s,r)
                s esta ocupado
                s' esta libre
    Retornar M
    
\end{lstlisting}    

\subsection{Alternativas}
\subsubsection{Diferentes cantidades de oferentes que requeridos}

Dado n oferentes y m requeridos, con \(m <> n\), no se puede encontrar un matching stable. 

Entonces, tenemos que redefinir el concepto de estable. Una pareja (s,r) es \textbf{estable} si:
\begin{itemize}
    \item No existe requerido r' sin pareja al que s prefiera a su actual pareja.
    \item No existe un requerido r' en pareja, tal que s y r' se prefieran sobre sus respectivas parejas.
    \item No existe solicitante s' sin pareja al que r prefiera a su actual pareja.
    \item No existe un solicitante s' en pareja tal que r y s' se prefieran sobre sus respectivas parejas.
\end{itemize}    


Por lo tanto un matching es estable si:
\begin{itemize}
    \item No tienen parejas inestables bajo la condicion anterior.
    \item Que no queden requeridos y solicitantes sin pareja.
\end{itemize}

Soluciones para ajustar al modelo de Gale-Shapley:
\begin{enumerate}
    \item Inventar \(|n-m|\) elementos ficticios
    \begin{itemize}
        \item Los elementos ficticios se pondran en las listas de preferencias con menos elementos.
        \item Estos elementos ficticios se agregan al final y deben ser los menos preferidos.
        \item Luego ejecutar Gale-Shapley
        \item Por ultimo, eliminar las parejas con elementos ficticios. Estos seran los requeridos que quedan sin pareja.
    \end{itemize}
    \item Adecuar el Algoritmo
    \begin{itemize}
        \item Si hay mas \textbf{solicitantes} que requeridos, quitar de la \textit{lista de solicitantes} sin parejas a aquellos que agotaron sus propuestas.
        \item Si hay mas \textbf{requeridos} que solicitantes, quitar de la \textit{lista de parejas} a aquellas donde el requerido quedo sin pareja.
    \end{itemize}
\end{enumerate}


\subsubsection{Preferencias incompletas}
Las listas de preferencias de los oferentes y los requeridos son un subset de las contrapartes.

Son parejas \textbf{aceptables} de un elemento a aquellas contrapartes que figuran en su lista de preferencias.

Una pareja (s,r) es \textbf{estable} si:
\begin{itemize}
    \item Son \textit{aceptables} entre ellos.
    \item No existe requerido \textit{aceptable} r' sin pareja al que s prefiera a su actual pareja.
    \item No existe un requerido \textit{aceptable} r' en pareja, tal que s y r' se prefieran sobre sus respectivas parejas.
    \item No existe solicitante \textit{aceptable} s' sin pareja al que r prefiera a su actual pareja.
    \item No existe un solicitante \textit{aceptable} s' en pareja tal que r y s' se prefieran sobre sus respectivas parejas.
\end{itemize}

\begin{quote}
    \textbf{Un matching es estable si no tiene parejas inestables bajo la condicion anterios.}
\end{quote}

\begin{lstlisting}[language=Python, caption=Algoritmo para parejas incompletas]
Inicialmente M=Vacio

#Iterea mientras no haya acotado su sublista de preferencias
Mientras existe un solicitante sin pareja
                'que no aun se haya postulado a todas las parejas' 

    Sea s un solicitante sin pareja
    Sea r el requerido de su mayor preferencia al que no le
                solicito previamente
    
    # se condiera si es aceptable
    if r considera 'aceptable' a s

        if r esta desocupado
            M = M U (s,r)
            s esta ocupado
        else
            Sea s' tal que (s', r) pertenece a M
            si r prefiere a s sobres s'
                M = M - {(s', r)} U (s,r)
                s esta ocupado
                s' esta libre

# Retornar solo parejas aceptables
Retornar M

\end{lstlisting}    


\subsubsection{Preferencias con empates}


\textbf{INDIFERENCIA Y PREFERENCIA ESTRICTA}

\begin{enumerate}
    \item X es \textbf{indiferente} a "y" y a "z" si en su lista de preferencias estan el la misma posicion.
    \item X es \textbf{prefefiere estrictamente} a "y" sobre "z" si en su lista de preferencias no le son indiferentes y 
          "y" se encuentra antes que "z" en la misma.
\end{enumerate}

\noindent
\textbf{ESTABILIDAD DEBIL}
\newline Una pareja (s,r) es debilmente estable si no existe una pareja (s' y r') talque:
\begin{itemize}
    \item s prefiere estrictamente a r' sobre r \textit{(pareja actual de s)}
    \item r' prefiere estrictamente a s sobre s' \textit{(pareja actual de r')}
\end{itemize}



\begin{lstlisting}[language=Python, caption=Algoritmo para parejas incompletas]
    Inicialmente M=Vacio
    
    #Iterea mientras no haya acotado su sublista de preferencias
    Mientras existe un solicitante sin pareja
                    'que no aun se haya postulado a todas las parejas' 
    
        Sea s un solicitante sin pareja
        Sea r el requerido de su mayor preferencia al que no le
                    solicito previamente
            
        if r esta desocupado
            M = M U (s,r)
            s esta ocupado
        else
            Sea s' tal que (s', r) pertenece a M

            # prefiere estrictamente
            si r prefiere estrictamente a s sobres s'
                M = M - {(s', r)} U (s,r)
                s esta ocupado
                s' esta libre
    
    Retornar M
    
\end{lstlisting}    
\begin{quote}
    En caso de que sea empate, se mantendra con su pareja actual.
\end{quote}

\noindent
\textbf{ESTABILIDAD FUERTE}
\newline Una pareja (s,r) es debilmente estable si no existe una pareja (s' y r') talque:
\begin{itemize}
    \item s prefiere estrictamente o le es indiferente a r' sobre r \textit{(pareja actual de s)}
    \item r' prefiere estrictamente o le es indiferente a s sobre s' \textit{(pareja actual de r')}
\end{itemize}
Puede no existir un matching perfecto.

\begin{lstlisting}[language=Python, caption=Algoritmo para parejas super estables]
    Inicialmente M=Vacio
    
    Mientras existe un solicitante sin pareja y no exista solicitante que agoto sus parejas
    
        Sea s un solicitante sin pareja
        Sea r el requerido de su mayor preferencia al que pueda proponer
        Por cada sucesor s' a s en la lista de preferencias de r
            if (s',r) pertence a M
                M = M - {(s',r)}
                s' esta libre
            quitar s' de la lista de preferencias de r
            quitar r de la lista de preferncias de s'

        Por cada requerido r' que tiene multiples parejas
            Por cada pareja s' en pareja con r' 
                M = M - {(s',r')}
                quitar s' de la lista de preferencias de r'
                quitar r' de la lista de preferencias de s'

    if estan todos en pareja
        Retornar M
    else
        No existe ningun matching super estable
\end{lstlisting}    
\begin{quote}
    En caso de que sea empate, se mantendra con su pareja actual.
\end{quote}

\subsubsection{Agrupacion de 1 a muchos}
El solicitante puede tener varios cupos por lo tanto:
\begin{itemize}
    \item Exiten \(m\) requeridos, donde un requerido puede estar unicamente con 1 pareja.
    \item Exiten \(n\) solicitantes, donde cada solicitante puede tener \(c\) cupos para armar parejas.
\end{itemize}

Existe un matching estable si la cantidad de requeridos es igual a la cantidad de solicitantes por la cantidad de cupos.

\begin{equation} \label{eu_eqn}
    m=n*c
\end{equation}

No cambia la definición de Gale Shampey para \textbf{matching estable}

\begin{lstlisting}[language=Python, caption=Algoritmo de solicitantes con cupos]
    Inicialmente M=Vacio
    
    Mientras exista un solicitante con cupo disponible
    
        Sea s un solicitante sin pareja
        Sea r el requerido de su mayor preferencia al que no le
                    solicito previamente
            
        if r esta desocupado
            M = M U (s,r)
            s decremente su disponibilidad de parejas
        else
            Sea s' tal que (s', r) pertenece a M

            si r prefiere a s sobres s'
                M = M - {(s', r)} U (s,r)
                s decremente su disponibilidad de parejas
                s' incrementa su disponibilidad de parejas
    Retornar M
    
\end{lstlisting}    
\begin{quote}
    \textbf{La complejidad algoritmica no se modifica porque solo se agrega un contador.}
\end{quote}


\subsubsection{Agrupacion de muchos a 1}
El requerido puede tener varios cupos por lo tanto:
\begin{itemize}
    \item Exiten \(m\) requeridos, donde cada solicitante puede tener \(z\) cupos para armar parejas.
    \item Exiten \(n\) solicitantes, donde un requerido puede estar unicamente con 1 pareja.
\end{itemize}

Existe un matching estable si la cantidad de solicitantes es igual a la cantidad de requeridos por la cantidad de cupos.

\begin{equation} \label{eu_eqn}
    n=m*z
\end{equation}

No cambia la definición de Gale Shampey para \textbf{matching estable}

\begin{lstlisting}[language=Python, caption=Algoritmo de requeridos con cupos]
    Inicialmente M=Vacio
    
    Mientras exista un solicitante con cupo disponible
    
        Sea s un solicitante sin pareja
        Sea r el requerido de su mayor preferencia al que no le
                    solicito previamente
            
        if r tiene cupo 
            M = M U (s,r)
            s esta ocupado
            r decrementa su disponibilidad de parejas
        else
            Sea s' tal que (s', r) pertenece a M y 
                    s' es el menos preferidos de las parejas r

            si r prefiere a s sobres s'
                M = M - {(s', r)} U (s,r)
                s esta ocupado
                s' esta libre
    Retornar M
    
\end{lstlisting}    
\begin{quote}
    \textbf{La complejidad algoritmica si se modifica.}
\end{quote}
Para conocer el solicitante de menor preferencia podemos utilizar un heap de minimos. Como el cupo es de z, la complejidad algoritmica para actualizar el heap es \(log (z)\).

\subsubsection{Agrupacion de y a x}
\begin{itemize}
    \item Exiten \(n\) solicitantes, donde cada solicitante puede tener \(c\) cupos para armar parejas.
    \item Exiten \(m\) requeridos, donde cada requerido puede tener \(z\) cupos para armar parejas.
\end{itemize}


Existe un matching estable si:

\begin{equation} \label{eu_eqn}
    n*c=m*z
\end{equation}


No cambia la definición de Gale Shampey para \textbf{matching estable}
\newline
Para implementar se requieren las siguientes estructuras:
\begin{itemize}
    \item Un heap de minimos para los requeridos.
    \item Un contador de cupos para los solicitantes.
\end{itemize}

\begin{quote}
    \textbf{La complejidad algoritmica es igual a la de los requeridos con cupos}
\end{quote}

\subsubsection{Conjuntos no bipartios - Stable Roommate Problem}
Pendiente

\section{Analisis amortizado}

\subsubsection{Metodo de agregacion}
\subsubsection{Metodo del banquero}
\subsubsection{Metodo del potencial}

\subsubsection{Heap binomial y fibonacci}
Revisar capitulo 19 del Corven.
\newline
Para el \textbf{heap binomial} se utilizan bosques de arboles binarios. Existe un proceso donde se van ordenando los arboles.

Al insertar, se parece al ejemplo de contador binario y la amortizacion es O(1)
\newline
Decrementar en un log binomial, es log(n) porque no es posible amortizar
\newline 
Eliminar el minimo, es el el peor caso es log(n)

Para el \textbf{heap fibonacci} ...

\section{Algoritmos Greedy}

Utiliza heurisica de seleccion para encontrar una solución global optima despues de muchos pasos.

\subsubsection{Mochila fraccionaria}

Dado un contener de capacidad W, y un conjunto de elementos n fraccionables de valor \(v_i\) y peso \(w_i\)

El objetivo es seleccionar un subconjunto de elemento o fracciones de ellos de modo de maximizar el valor almacenado y sin superar la capacidad de la mochila.

La complejidad es \(O(nlog(n))\)

\subsubsection{Cambio de moneda}

Es una solución es conocido como solución de cajero. Contamos con un conjunto de diferentes monedas de diferentes denominación sin restricción de cantidad.

\[
    \$=(C_1,C_2,C_3,\cdots,C_n)  
\]

El objetivo es entregar la menor cantidad posible de monedas como cambio.

Tiene una complejidad de \(O(n)\).

El sistema \(\$\) se conoce como \textbf{canonico} a aquel en el que para todo x, \(greedy(\$,x)=optimo(\$,x)\).

Para saber si una base es canonica:
\begin{enumerate}
    \item Basta con buscar un contraejemplo. Estaria entre la 3ra denomininacion y la suma de las ultimas dos doniminaciones.
    \item Utilizar un algoritmo Polinimico para determinar si es un sistema canonico.
\end{enumerate}

Si el problema no es greddy, se puede construir un algoritmo utilizando programación dinamica.


\subsubsection{Seam Carving}
Es un algoritmo para adecuar imagenes. Analiza imagenes recortando pixeles de menor importancia. Retira tantas vetas como sea necesario para llegar a un tamaño optimo.

\end{document}
