\documentclass{article}
\usepackage{amsmath,amssymb}
\usepackage{graphicx}
\usepackage[top=3cm, bottom=3cm, right=3cm, left=3cm]{geometry}

\begin{document}

\newpage{}
\tableofcontents
\newpage{}

\newpage
\section{Stable Maching problema}

\subsection{Algoritmo Gale-Shapley}
\subsection{Alternativas}
\subsubsection{Diferentes cantidades de oferentes que requeridos}

Dado n oferentes y m requeridos, con \(m <> n\), no se puede encontrar un matching stable. 

Entonces, tenemos que redefinir el concepto de estable. Una pareja (s,r) es \textbf{estable} si:
\begin{itemize}
    \item No existe requerido r' sin pareja al que s prefiera a su actual pareja.
    \item No existe un requerido r' en pareja, tal que s y r' se prefieran sobre sus respectivas parejas.
    \item No existe solicitante s' sin pareja al que r prefiera a su actual pareja.
    \item No existe un solicitante s' en pareja tal que r y s' se prefieran sobre sus respectivas parejas.
\end{itemize}    


Por lo tanto un matching es estable si:
\begin{itemize}
    \item No tienen parejas inestables bajo la condicion anterior.
    \item Que no queden requeridos y solicitantes sin pareja.
\end{itemize}

Soluciones para ajustar al modelo de Gale-Shapley:
\begin{enumerate}
    \item Inventar \(|n-m|\) elementos ficticios
    \begin{itemize}
        \item Los elementos ficticios se pondran en las listas de preferencias con menos elementos.
        \item Estos elementos ficticios se agregan al final y deben ser los menos preferidos.
        \item Luego ejecutar Gale-Shapley
        \item Por ultimo, eliminar las parejas con elementos ficticios. Estos seran los requeridos que quedan sin pareja.
    \end{itemize}
    \item Adecuar el Algoritmo
    \begin{itemize}
        \item Si hay mas \textbf{solicitantes} que requeridos, quitar de la \textit{lista de solicitantes} sin parejas a aquellos que agotaron sus propuestas.
        \item Si hay mas \textbf{requeridos} que solicitantes, quitar de la \textit{lista de parejas} a aquellas donde el requerido quedo sin pareja.
    \end{itemize}
\end{enumerate}


\subsubsection{Preferencias incompletas}
Las listas de preferencias de los oferentes y los requeridos son un subset de las contrapartes.

Son parejas \textbf{aceptables} de un elemento a aquellas contrapartes que figuran en su lista de preferencias.

Una pareja (s,r) es \textbf{estable} si:
\begin{itemize}
    \item Son \textit{aceptables} entre ellos.
    \item No existe requerido \textit{aceptable} r' sin pareja al que s prefiera a su actual pareja.
    \item No existe un requerido \textit{aceptable} r' en pareja, tal que s y r' se prefieran sobre sus respectivas parejas.
    \item No existe solicitante \textit{aceptable} s' sin pareja al que r prefiera a su actual pareja.
    \item No existe un solicitante \textit{aceptable} s' en pareja tal que r y s' se prefieran sobre sus respectivas parejas.
\end{itemize}

\begin{quote}
    \textbf{Un matching es estable si no tiene parejas inestables bajo la condicion anterios.}
\end{quote}



\subsubsection{Preferencias con empates}

\end{document}

